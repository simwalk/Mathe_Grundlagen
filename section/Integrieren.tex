%%%%%%%%%%%%%%%%%%%%%%%%%%%%%%%%%%%%%%%%%%%%%%%%%%%%%%%%%%%%%%%%%%%%%%%%%%%%%%%%%%%%%%%%%%%%%%%%
% Integral 
%%%%%%%%%%%%%%%%%%%%%%%%%%%%%%%%%%%%%%%%%%%%%%%%%%%%%%%%%%%%%%%%%%%%%%%%%%%%%%%%%%%%%%%%%%%%%%%%

\section{Integrieren}

\subsection{Wozu brauche ich die Integralrechnung?}
Die Integralrechnung ist neben der Differentialgleichung der wichtigste Zweig der mathematischen Disziplin Analysis. Sie ist aus dem Problem der Flächen- und Volumenberechnung entstanden. (Quelle Wikipedia)

\subsection{Stammfunktion}
Jede auf $[a,b]$ differenzierbare Funktion $F$ nennt man Stammfunktion, wenn $F'= f$.\\
Umgangssprachlich nennt man dies \textbf{\grqq aufleiten\grqq} (Gegenteil von ableiten)

\subsection{Leibniz- oder Integralschreibweise}
\label{Leibniz}
$\displaystyle I = \int\limits_{a}^{b}{f(x)}d{x} = \bigl[F(x)\bigr]_a^b = F(b)-F(a)$

\subsection{Bestimmtes Integral}
Ein bestimmtes Integral besitzt obere und untere Grenzen. Die Integrationskonstante $C$ verschwindet.\\
\verweis{Leibniz}{Leibniz- oder Integralschreibweise}

\subsection{Unbestimmtes Integral}
Im Unterschied zu einem bestimmten Integral besitzt ein unbestimmtes Integral \textbf{keine} Grenzen.\\\\
$\displaystyle I = \int{f(x)}d{x} = F(x) + C$

\subsection{Wie löse ich ein Integral?}
Sobald das Integral in der Leibnizform steht, gehe ich wie folgt vor:
\subsubsection{Bestimmtes Integral}
\begin{compactenum}
	\item Stammfunktion der Funktion $f(x)$ bilden
	\item Nacheinander die beiden Grenzen $b$ und $a$ in die Stammfunktion einsetzen
	\item Lösung des Integrals: Die beiden eingesetzten Grenzen voneinander subtrahieren. $I=F(b)-F(a)$
\end{compactenum}
\subsubsection{Unbestimmtes Integral}
\begin{compactenum}
	\item Stammfunktion der Funktion $f(x)$ bilden
	\item Lösung des Integrals: Stammfunktion + Integrationskonstante $C$
\end{compactenum}

\subsection{Wichtige Integrale}
\renewcommand{\arraystretch}{2.0}
\begin{tabular}{|c|c|c|c|c|c|c|c|}
	\cline{1-2}\cline{4-5}\cline{7-8}
	\boldmath${f(x)}$ & \boldmath${F(x)}$ &\qquad\qquad\qquad& \boldmath${f(x)}$ & \boldmath${F(x)}$ &\qquad\qquad\qquad& \boldmath${f(x)}$ & \boldmath${F(x)}$\\
	\cline{1-2}\cline{4-5}\cline{7-8}
	$\displaystyle \int{0}dx$ & $C$ && $\displaystyle \int{x^n}d{x}$ & $\dfrac{x^{n+1}}{n+1}+C \qquad (n\ne-1)$ && $\displaystyle \int{sin(x)}dx$ & $-cos(x)+C$\\
	\cline{1-2}\cline{4-5}\cline{7-8}
	$\displaystyle \int{1}dx$ & $x+C$ && $\displaystyle \int{\dfrac{1}{x}}dx$ & $ln|x|+C$ && $\displaystyle \int{cos(x)}dx$ & $sin(x)+C$\\
	\cline{1-2}\cline{4-5}\cline{7-8}
	$\displaystyle \int{e^x}dx$ & $e^x+C$ && $\displaystyle \int{ln(x)}dx$ & $-x+x\cdot ln(x)+C$ && $\displaystyle \int{tan(x)}dx$ & $-ln(|cos(x)|)+C$\\
	\cline{1-2}\cline{4-5}\cline{7-8}
\end{tabular}
\renewcommand{\arraystretch}{1.0}

\subsection{Rechenregeln}
\begin{tabular}{l c}
	$\int\limits_a^b f({x}) d{x} = F(b) - F(a)$ & $|\int\limits_a^b f(x) dx| \leq \int\limits_a^b |f({x})| d{x}$\\
	$\int\limits_a^b f({x}) d{x} = \int\limits_0^b f({x}) d{x} - \int\limits_0^a f({x})d{x} = \int\limits_0^b f({x}) d{x} - (-1) \cdot \int\limits_a^0 f({x})d{x}$
\end{tabular}

\subsection{Partielle Integration}
Sobald ich ein Produkt zweier Funktionen habe und davon das Integral berechnen möchte, bietet sich die \textbf{\grqq Partielle Integration\grqq} an.\\
Wie gehe ich vor:
\begin{compactenum}
	\item Von der einen Funktion die Stammfunktion bilden
	\item Multiplikation mit der anderen Funktion
	\item Das Produkt wird subtrahiert mit dem Integral von der Stammfunktion mal die Ableitung der anderen Funktion\\
\end{compactenum}
\begin{minipage}[t]{1cm}
	\textbf{Bsp.:}
\end{minipage}
\begin{minipage}[t]{16cm}
	$\displaystyle \int x\cdot sin(x) dx = \int f(x) \cdot g'(x) dx = f(x) \cdot g(x) - \int f'(x) \cdot g(x) dx = -x \cdot cos(x) - \int 1 \cdot (-cos(x)) dx = \underline{\underline{-x \cdot cos(x) + sin(x) + C}}$
\end{minipage}

\subsection{Substitution}
Hierzu ein erklärendes Beispiel:\\\\
\begin{minipage}[t]{1cm}
	\textbf{Bsp.:}
\end{minipage}
\begin{minipage}[t]{16cm}
	$\displaystyle f(x) = 2x \cdot ln(x^2) \Rightarrow F(x) = \int_{1}^{2} 2x \cdot ln(x^2) dx$\\
	Lösung des Integrals durch Substitution:
	\begin{compactenum}
		\item Substitution \qquad $\displaystyle u(x) = x^2$
		\item $dx$ ersetzen \qquad $\displaystyle u'(x) = \dfrac{du}{dx}=2x \Rightarrow dx = \dfrac{1}{2x}du$
		\item Substitution der Grenzen:\\
		- untere Grenze: \qquad $u(1) = 1$\\
		- obere Grenze: \qquad $u(2) = 4$
		\item In das Integral einsetzen und auflösen:\\
		$\displaystyle \int_{1}^{4} 2x \cdot ln(u) \cdot \dfrac{1}{2x}du = \int_{1}^{4} ln(u) du= \bigl[u \cdot ln(u)-u\bigr]_1^4 = \bigl[4 \cdot ln(4)-4\bigr] - \bigl[1 \cdot ln(1)-1\bigr] \approx \underline{\underline{2.545}}$
	\end{compactenum}
\end{minipage}

\subsection{Mittelwerte mittels Integral berechnen}
\begin{tabular}{ll}
	\textbf{Linearer Mittelwert} &
	\textbf{Quadratischer Mittelwert}\\
	$\bar{f} = \frac{1}{b-a} \int\limits_{a}^{b} f(x)dx$ &
	$\bar{f} = \sqrt{\frac{1}{b-a} \int\limits_{a}^{b} f(x)^2dx}$
\end{tabular}

\input{tabellen/Integrale.tex}

